\documentclass{article}
\title{\emph{Summary of paper:}\\Adaptive background mixture models for real-time
tracking}
\author{Chiel Kooijman \& Auke Wiggers}
\usepackage{microtype}
\usepackage[round]{natbib}
\usepackage[scaled]{helvet}
\renewcommand*\familydefault{\sfdefault}
\usepackage[T1]{fontenc}
\begin{document}
\maketitle

\section{Introduction}
\subsection{Shortcomings of previous method}
	Averaging images over time:
	\begin{itemize}
		\item Not robust to many (especially slowly) moving objects
		\item Difficulty with bimodal backgrounds (I think this means existing
		of two very different looking components)
		\item Recovers slowly when the background is uncovered (?)
		\item Has a single predetermined threshold
	\end{itemize}
	Other solution (Riddler et al.) uses kalman filter to model pixels, which
	has a pixel-wise automatic threshold, but still:
	\begin{itemize}
	\item Recovers slowly
	\item Does not handle bimodal backgrounds
	\end{itemize}

	Pfinder uses a ``multi-class statistical model'' for tracked objects with a
	Gaussian per background pixel, which works well after an initialization
	period in which nothing moves.

	Friedmann and Russell use a pixel-wise EM algorithm that classifies to
	three groups: Car, Road, Shadow


\subsection{Approach proposed in this paper}
	Model the values of a pixel as a mixture of Gaussians. Pixels that do not
	fit in the background distribution are considered foreground.

	There are two significant parameters:
	\begin{itemize}
	\item Learning constant $\alpha$
	\item Proportion of data that should be background $T$
	\end{itemize}

\section{Method}
	\emph{Multiple} Gaussians represent multiple surfaces in the background\\
	\emph{Adaptive} Gaussians model the changing lighting conditions

	Pixels that do not match their ``background Gaussians'' are grouped using
	connected components and tracked using a multiple hypothesis tracker.

	We consider a ``pixel process'' the history of its values over time.
	$$\{X_1, \ldots, X_t \} = \{I(x_0, y_0, i) : 1 \leq i \leq t\}$$

	Assumptions:
	\begin{itemize}
		\item More recent observations are more important determining Gaussian,
			because otherwise newly added objects would not be considered
			background until they had been there longer than they had not been
			there, which could be an arbitrarily long time
		\item Moving objects have more variance than static objects
		\item There should be more data supporting the background distribution
			because they are are repeated (?), and different objects are often
			differently coloured.
	\end{itemize}

	The history of each pixel is modelled as a mixture of $K$ Gaussians. The
	probability of observing the current pixel value is:
	$$P(X_t) = \sum^K_{i=1} \omega_{i,t}~\eta
	(X_i, \eta_{i, t}, \Sigma_{i,t})$$
	$\omega$ is an estimate of the weight (proportion) of the Gaussian.\\
	$\eta$ is the Gaussian PDF.

	(The authors note that they use the assumption that Red, Green, and Blue
	channels are independent and have the same variance---by using a diagonal
	covariance matrix---which speeds up computation time at the cost of
	accuracy.)

	This method can be seen as expectation maximization, treating new
	observations as sample sets of size 1, because it is non-stationary.

	We use on-line K-means because calculating exact EM for each pixel is too
	costly.

	Every new pixel value, $X_t$, is checked against the existing $K$ Gaussian
	distributions, until a match is found. A match is defined as a pixel value
	within 2.5 standard deviations of a distribution.
	If none of the $K$ distributions match, the least probable distribution is
	replaced with $\mu$ as the current value and a high variance, and weight:
	$$\omega_{k,t} = (1 - \alpha) \omega_{k,t-1} + \alpha(M_{k,t})$$


\section{Questions}
	\begin{itemize}
		\item Does this algorithm perform well if the camera is moving? And if
			not, how can this be remedied?
		\item Would it be more accurate to calculate the probability of a pixel
			given \emph{all} mixtures instead of only the most likely one? This
			would result in a probability instead of a hard classification to
			the foreground or background. Would a probabilistic model yield
			extra benefits for object tracking?
		\item Is it possible with modern hardware to use a real covariance
			matrix for the mixture models, and would this improve performance?
	\end{itemize}
\end{document}
% vim: set spell:
